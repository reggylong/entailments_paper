
\section{Pipeline}
We describe the pipeline $p$ that we run on each text $T$. The pipeline
consists of multiple natural language annotators that we run in 
the sequence outlined below.
\citep{manning2014stanford}

\subsection{Tokenizer}
The tokenizer takes the utterances $u_{1:n}$ and outputs a tuple of
each word in $u_{1:n}$. For example, the tokenizer would transform
the sentence \nl{He'll travel to Minneapolis} to the tuple
\begin{center}
  (\nl{He}, \nl{'ll}, \nl{travel}, \nl{to}, \nl{Minneapolis})
\end{center}
\subsection{Sentence Splitter}
Given the tokens of the utterances $u_{1:n}$, the sentence splitter outputs
a tuple of the sentences $(u_1,\dots,u_n)$. This is necessary
for the tasks below because they operate at the sentence level.
\subsection{Part of Speech}

\begin{figure}
\includegraphics[scale=0.33]{figures/pos.png}
\caption{
\label{fig:pos}
Part of Speech tags on example sentence.
}
\end{figure}

\citet{toutanova2003tagger}
\subsection{Named Entity Recognition}

\begin{figure}
\includegraphics[scale=0.33]{figures/ner.png}
\caption{
\label{fig:ner}
Named Entity Recognition on example sentence.
}
\end{figure}

\citet{finkel2005incorporating}
\subsection{Dependency Parse}

\begin{figure}
\includegraphics[scale=0.33]{figures/dep.png}
\caption{
\label{fig:dep}
Dependency parse on example sentence.
}
\end{figure}

\citet{chen2014nndep}
\subsection{Mention Resolver}

\subsection{Natural Logic Resolver}
\subsection{Coreference Resolution}

\begin{figure}
\includegraphics[scale=0.33]{figures/coref.png}
\caption{
\label{fig:coref}
Coreference resolution on example sentence.
}
\end{figure}

\citet{clark2015coref}

\section{Relation Extractor}
Limit max number of entailments taken per clause to 100.
\citet{angeli2015openie}
\citet{fader11reverb}

\subsection{With Coreference Resolution}

\subsection{Without Coreference Resolution}
