\section{Experiments}

Our experiments target the following questions
\begin{enumerate}
\item Is it computationally feasible to extract relations from 
  such a large corpus?
\item How does our system compare to the state of the art (e.g. \reverb).
\item Does coreference resolution improve the number of 
  ``usable'' relation triples?
\end{enumerate}

\subsection{Dataset}
We use the \newsspike{} dataset collected in \citet{zhang2013parallelparaphrase}.
This dataset is a collection of news article texts, all taken from the months of
Janurary and February in 2013, and contains 546,713 such texts. On average,
each news article contains between 31 and 32 sentences.

\subsection{Results}


\input tables/resources.tex
\input tables/raw_counts.tex 
\input figures/extrapolated.tex

\input figures/pronouns.tex

\input figures/triples.tex
\input tables/pronoun_adjusted.tex

\paragraph{Computational Feasibility.}
For each of the three types, we time out
the annotation/extraction process after no more than
2 minutes. As a result, the three systems failed to
annotate and extract relations from 
a significant number (10-30k) of articles. Compared to
\reverb{} and \openie{}, \openiecoref{} requires a much 
larger amount of memory to run effectively. For
example, reducing the amount of memory from 128Gb to
32Gb increases the runtime by more than 4 times.

\paragraph{\openie{} vs. \reverb.}
We compare the \openie{} pipeline described above
against the \reverb{} pipeline. The \reverb{} system uses 
a separate pipeline, which includes a tokenizer, POS tagger,
and sentence chunker (similar to a constituency parse, but 
more limited). 
We refer to \citet{fader11reverb} for
the specifics of the \reverb{} system.

\reffig{extrapolated} shows that
the two \openie{} systems extract three times
as many relations per sentence than \reverb{}. Importantly,
as \reffig{triples} shows, the number of unique relations 
remains relatively the same between all three systems, which
means that \openie{} obtains many more extractions per
relation than \reverb{}.
This suggests 
that the \openie{} systems are more efficient. One avenue
of future work is to investigate quantitative measures for
the amount of noise in the extracted relations between \openie{}
and \reverb{}.

\paragraph{Coreference Resolution.}
Intuitively, using coreference resolution should reduce the number
of relation triples that contain pronouns. \reffig{pronouns} shows
that this is indeed the case; \openiecoref{} halves the percentage
of relation triples with pronouns. See \reftab{pronoun_adjusted} for the
number of relation triples extracted that don't contain pronouns.
Although \openiecoref{} annotated 20k fewer articles, it was able
to extract the most number of usable relations.
