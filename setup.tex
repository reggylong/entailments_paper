\section{Setup}
This section formalizes the notion of an entailment graph, and describes
the notation used in the rest of the paper.
\subsection{Notation}
Let $u_{1:n}$ denote a concatenation of utterances, and 
$d$ denote the date that an article was written. We define
an text $T \equiv (u_{1:n}, d)$. For example, a text could consist
of the following:
\begin{description}

  \item $u_1 =$ \nl{The hearing came as the White House announced that President Obama will travel to Minneapolis on Monday to meet with local leaders and law enforcement officials about his newly unveiled gun control plans.}
  \item $u_2 =$ \nl{White House spokeswoman Joanna Rosholm said that `Minneapolis is a city that has taken important steps to reduce gun violence and foster a conversation in the community about what further action is needed'.}
  \item $d =$  \nl{January 31, 2013}
  \item $T = (u_{1:2}, d)$
\end{description}

We apply a pipeline $p$ of natural language annotators 
to every utterance in $u_{1:n}$ in a text $T$. 
The pipeline $p$ takes as input an article, and outputs an annotation $A$.
An annotation $A$ is a set where each element contains
information about the the text $T$. For example, if our
pipeline consisted of a sentence splitter, then calling $p$
on the example above would produce:
\begin{center}
  $p(T) = A = \{\text{Sent 1: } u_1, \text{Sent 2: } u_2\}$.
\end{center}

Finally, we use a relation extractor $e$ that takes as input
the text $T$ and annotation $A$, and outputs a set of relation triples $R=\{r_1,\dots,r_m\}$,
where each relation triple $r$ is a tuple $r=$(\wl{subect}, \wl{relation}, \wl{object}).
For example, one such relation triple from the aforementioned text is the following
\begin{center}
  $r$ = (\nl{Obama}, \nl{will travel to}, \nl{Minneapolis})
\end{center}

\subsection{Entailment Graphs}
An entailment graph $G=(V,E)$ is a directed graph where the vertices $V$
are relations, and the edge set $E$ indicates the entailments between
relations.
\input figures/graph
\reffig{graph} shows an example entailment graph with the three relations:
\textsc{visit}, \textsc{travel}, and \textsc{leave}. 
To visit a location implies the travel relation,
and to leave a location implies that one previously visited/traveled to that
location. Each vertex is typed (e.g. \textsc{visit} is a relation between
a Person and a Location), and edges can only connect two vertices with
the same type.

\subsection{Problem Statement}
Given a set of $N$ texts, $\{T_i\}_{i=1}^{N}$, generate a set of 
relation triples $R$. Then learn a globally consistent mapping
from relation triples to an entailment graph $G$. This paper will focus
on the first half of the problem.


