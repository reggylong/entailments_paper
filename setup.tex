\section{Setup}
This section formalizes the notion of an entailment graph, and describes
the notation used in the rest of the paper.
\subsection{Notation}
Let $u_{1:n}$ denote a concatenation of utterances, and 
$d$ denote the date that an article was written. We define
an text $T \equiv (u_{1:n}, d)$. For example, a text could consist
of the following:
\begin{description}
  \item $u_1 =$ President Obama heads to Minneapolis today to push his message of reducing gun violence.
  \item $u_2 =$ It comes after a weekend where we saw more devastating gun deaths.
  \item $d =$  February 4, 2013
  \item $T = (u_{1:2}, d)$
\end{description}

We apply a pipeline $p$ of natural language annotators 
to every utterance in $u_{1:n}$ in a text $T$. 
The pipeline $p$ takes as input an article, and outputs an annotation $A$.
An annotation $A$ is a set where each element contains
information about the the text $T$. For example, if our
pipeline consisted of a sentence splitter, then calling $p$
on the example above would produce:
\begin{center}
  $p(T) = A = \{\text{Sent 1: } u_1, \text{Sent 2: } u_2\}$.
\end{center}

Finally, we use a relation extractor $e$ that takes as input
the text $T$ and annotation $A$, and outputs a set of relations $R=\{r_1,\dots,r_m\}$.

\subsection{Problem Statement}
Given a set of $N$ texts, $\{T_i\}_{i=1}^{N}$, generate a set of 
relations $R$. Then learn a mapping
from relations to an entailment graph $G$ that maximizes the number of
correct entailments and minimizes the number of incorrect entailments.

